\documentclass[12pt,a4paper]{report}
\usepackage[utf8]{inputenc}
\usepackage[spanish]{babel}
\usepackage{amsmath}
\usepackage{amsfonts}
\usepackage{amssymb}
\usepackage{graphicx}
\usepackage[document]{ragged2e}
\usepackage[table]{xcolor}
\usepackage[left=2cm,right=2cm,top=2cm,bottom=2cm]{geometry}
\begin{document}
\begin{center}
			%%%
\newcommand{\HRule}{\rule{\linewidth}{0.5mm}}
\begin{minipage}{0.48\textwidth} \begin{flushleft}
\includegraphics[scale=0.20]{../../../../Downloads/logo_izquierda.png} 
\end{flushleft}\end{minipage}
\begin{minipage}{0.48\textwidth} \begin{flushright}
\includegraphics[scale=0.6]{../../../../Downloads/LOGO_DERECHA.png} 
\end{flushright}\end{minipage}
															%%%
\vspace*{1.0cm}	

\textsc{\huge Tecnológico Nacional de México\\ \vspace{5px} Campus Tapachula}\\[1.5cm]	

\textsc{\LARGE ALGEBRA LINEAL \\}
\vspace{2cm}										

\begin{minipage}{0.9\textwidth} 
\begin{center}	
		%%%
\textsc{\LARGE Problemario 1 }
\end{center}
\end{minipage}\\[0.5cm]
%%%
    																				%%%
 			\vspace*{1cm}																		%%%
																					%%%

 																					%%%
																	%%%
 																				%%%
																					%%%
\begin{minipage}{0.46\textwidth}													%%%
\begin{flushleft} \large															%%%

% Aqui a continuación pongan los nombres de los integrantes
\emph{Autores:}\\	
Kevin Darinel Salas Pérez\\
Haziel Antonio Romero Hernandez\\
Angel Fernando Morales Rodrigez\\
%%%
			%\vspace*{2cm}	
            													%%%
										 						%%%
\end{flushleft}																		%%%
\end{minipage}		
																%%%
\begin{minipage}{0.52\textwidth}		
\vspace{-0.6cm}											%%%
\begin{flushright} \large															%%%
\emph{Número de lista:} \\																	%%%
%\\Número %														%%%
\end{flushright}																	%%%
\end{minipage}	
\vspace*{1cm}
%\begin{flushleft}
 	
%\end{flushleft}
%%%
 																			%%%
\vspace{2cm} 																				
\begin{center}	
{\large \today}																	%%%
 			\end{center}												  						
\end{center}	

%%siguiente pagina%%%
\newpage
%%comienza pagina nueva%%%

\section*{Tema 1.2 Operaciones con números complejos.}
\justify 
Complete la table con los valores correctos correspondientes

\begin{tabular}{ | c | c | c | c | }
	\rowcolor{gray}
	Variable & conjugado & Operación & Resultado\\ 
	T = 2 + 2i  &  &T + Z \\
	\rowcolor{white}
	& &\\
 	\rowcolor{gray}
	Z= 3+i& & (2(W)-3(Q) / K)&\\
	\rowcolor{white}
	W= 1+ 3i& &T*Z&\\
	\rowcolor{gray}
	K= -3+2i& & (W*Q)-3(T)&\\
	\rowcolor {white}
	Q= -2i& & WQ-K(WQ-K(Q+W))&\\

	
\end{tabular}













\end{document}